\documentclass[a4paper]{article}
\usepackage[utf8]{inputenc}
\usepackage[english]{babel} 
\usepackage{hyperref} 
\usepackage{float}
\usepackage{amsmath} 
\usepackage{graphicx}
\usepackage[colorinlistoftodos]{todonotes} 
\usepackage{tikz}
\usepackage{pdfpages} 
\usepackage{setspace}
\usepackage{listings}
\usepackage[margin=0.5in]{geometry}
\usepackage{natbib}

\title {
	Annotations 2\\ Diet and social status in the Lejasbitēni Iron Age population from Latvia
}

\author {
	\normalsize Sathvik Chinta\\\normalsize
    \normalsize HSTAM 370\\\normalsize
}

\date {
	\color{black} October 28, 2022
}

\doublespacing
\begin{document} \maketitle
    \setcitestyle{authoryear,open={(},close={)}}

        \textbf{a paragraph (150-250 words) summarizing the author's main argument and the evidence they use to support that argument}
        
        The study, “Diet and social status in the Lejasbitēni Iron Age population from Latvia”, details isotope data from the 
        Lejasbitēni cemetery in Latvia. This burial ground was estimated to be from around the Iron Age between the 7th and 10th
        centuries CE. The main aim of the study was to see if there were significant differences between the populations of those
        that were buried at the cemetery. Namely, “if the development of the Viking Age brought about changes in social status as
        expressed by burial traditions and grave goods, and diet, in the Lejasbitēni population” \citep{PETERSONEGORDINA2022103519}. Their conclusion was that there 
        were clear differences in childhood diets between the sexes of those that were buried. However, they could find no 
        evidence to suggest that dietary differences within gender groups had any effect on the grave goods found alongside 
        the bodies, indicating that childhood diet had no significant impact on social status upon reaching adulthood. 
        Lastly, they saw “a change towards a more hierarchical society…in the later period of the cemetery” \citep{PETERSONEGORDINA2022103519}. 
        These findings were supported by archaeological data (for grave goods and burial traditions), 
        osteological data (to identify the sex and age of a skeleton), and carbon/nitrogen sample data 
        (to identify the diet of an individual). 

        \textbf{a brief paragraph (100-150 words) that analyzes, with at least one example, how the reading changes your understanding of another reading or lecture topic in this or another course you have taken or are taking at UW.}

        One of the biggest takeaways from this reading that really changed my perspective was the carbon and nitrogen sample data. 
        From physics, I learned about the half-life of carbon, and how scientist have used it to identify the age of many items. 
        What I wasn't aware of, however, was the fact that we could figure out the diet of an individual using the presence of 
        certain isotopes. The study says “Diet and nutritional status is recorded during the deposition of new bone and 
        other body tissues from where this information can be extracted by measuring $\delta^{13}C$ and 
        $\delta^{15}N$.”\citep{PETERSONEGORDINA2022103519} Furthermore, I was even more shocked to learn that ingesting foodstuffs based on 
        plants which photosynthesize is the reason for this carbon deposits on our bones (the researchers measured 
        the levels from teeth). 

        \pagebreak
        \bibliographystyle{apalike}
        \bibliography{myrefs}
        \cite{PETERSONEGORDINA2022103519}
\end{document}
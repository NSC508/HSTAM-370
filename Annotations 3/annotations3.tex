\documentclass[a4paper]{article}
\usepackage[utf8]{inputenc}
\usepackage[english]{babel} 
\usepackage{hyperref} 
\usepackage{float}
\usepackage{amsmath} 
\usepackage{graphicx}
\usepackage[colorinlistoftodos]{todonotes} 
\usepackage{tikz}
\usepackage{pdfpages} 
\usepackage{setspace}
\usepackage{listings}
\usepackage[margin=0.5in]{geometry}
\usepackage{natbib}

\title {
	Annotations 3\\ Isotopic provenancing of the Salme ship burials in Pre-Viking Age Estonia
}

\author {
	\normalsize Sathvik Chinta\\\normalsize
    \normalsize HSTAM 370\\\normalsize
}

\date {
	\color{black} November 11, 2022
}

\doublespacing
\begin{document} \maketitle
    \setcitestyle{authoryear,open={(},close={)}}

        \textbf{a paragraph (150-250 words) summarizing the author's main argument and the evidence they use to support that argument}
        
        This article deals with two ship burials that were found in Estonia from the Pre-Viking age. Being the first of its kind in Europe,
        this was a monumental discovery and the authors attempted to study the individuals found in the ship, and tried to answer questions
        such as class, origan, and other features of the inhabitants. They analyzed both the ships (Salme I and Salme II) and the 
        different items they found within. Salme I, for instance, had seven male skeletons along with an assortment of weapons, 
        game pieces, and animal remains. The Salme II ship was initially found with weapons, two human skeletons, and dog remains
        but additional excavations found 34 human remains. The researchers were then concerned with finding out where the 
        individuals on both ships originated from. They used isotopic analysis and examined Carbon, Strontium, and Oxygen 
        isotopes from dental enamel and compared these values with those found in various Nordic countries to narrow down 
        the location of the Salme ship men. They finally concluded that “the Malaren region in central Sweden [was] the most 
        probably homeland of those men who travelled to Salme, died violently and were buried hastily in two ships around AD 750” 
        \citep{PriceT.Douglas2016Ipot}. 

        \textbf{a brief paragraph (100-150 words) that analyzes, with at least one example, how the reading changes your understanding of another reading or lecture topic in this or another course you have taken or are taking at UW.}

        Previously, I learned that Carbon and Nitrogen isotopes in the teeth provided information about the diet of an individual
        in addition to figuring out their class standing in society by the amount of nutrition. However, this article also uses
        Oxygen as an isotope to identify certain features in the individuals found in the Salme ships. I learned that Oxygen 
        isotopes can reflect body water, and “ultimately drinking water…which in turn predominantly reflects local rainfall” 
        \citep{PriceT.Douglas2016Ipot}. The investigators used this in information to assist their search for where the individuals from 
        the Salme ships were from. Strontium isotopes were also examined, which I found interesting because it is not a common
        element in foods/water. Instead, it is present in “different kinds of rocks…[and] moves into humans from rocks and
        sediment through the food chain.” \citep{PriceT.Douglas2016Ipot}. These were also used in order to understand the geography 
        of the men from the Salme ships. 

        \pagebreak
        \bibliographystyle{apalike}
        \bibliography{myrefs}
        \cite{PriceT.Douglas2016Ipot}
\end{document}
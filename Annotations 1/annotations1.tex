\documentclass[a4paper]{article}
\usepackage[utf8]{inputenc}
\usepackage[english]{babel} 
\usepackage{hyperref} 
\usepackage{float}
\usepackage{amsmath} 
\usepackage{graphicx}
\usepackage[colorinlistoftodos]{todonotes} 
\usepackage{tikz}
\usepackage{pdfpages} 
\usepackage{listings}
\usepackage[margin=0.5in]{geometry}
\usepackage{natbib}

\title {
	Annotations 1
}

\author {
	\normalsize Sathvik Chinta\\\normalsize
    \normalsize HSTAM 370\\\normalsize
}

\date {
	\color{black} October 15, 2022
}


\begin{document} \maketitle
    \setcitestyle{authoryear,open={(},close={)}}
    \section{}
        \textbf{At the threshold of the Viking Age: New dendrochronological dates for the Kvalsund ship and boat bog offerings (Norway)}

        \textbf{a paragraph (150-250 words) summarizing the author's main argument and the evidence they use to support that argument}
        This article concerns the dating of two Viking ships found at a burial site in Kvalsund.
        Despite being excavated a very long time ago, the exact dating of the ships has been a topic of 
        debate for a long time. The article provides background to the find, as well as proposing new dates for the 
        Kvalsund ships based on dendrochronological analysis. Initially radiocarbon dated, the ships 
        were initally thought to be from the 7th-8th centuries. And while "most ships from the Late Iron 
        Age in Norway (i.e. c. AD 560-1050) are provided with a more accurate date by 
        dendrochronology, but the Kvalsud vessels have lacked dendrochronological dates." \citep{NORDEIDE2020102192}. 

        After descriptions of the burial site and it's historical significance, the article goes on to explain how they used 
        cross-section sampling in order to re-date the ships. Upon utilizing this method, the 
        ships were estimated to be from the end of the 8th century, "thus dating the vessels at the threshold of the Viking Age"
        \citep{NORDEIDE2020102192}.
        
        \textbf{a brief paragraph (100-150 words) that analyzes, with at least one example, how the reading changes your understanding of another reading or lecture topic in this or another course you have taken or are taking at UW.}
        
        This article was very interesting to me because up until now I had only heard of carbon dating. In class, we 
        discussed the emergence of sails on Viking ships. We saw that sails first began to emerge around the year
        650. However, the Kvalsund ships (from a hunred years later) "was not yet fully developed as a sailing ship" \citep{NORDEIDE2020102192}.
        This was surprising to me, since I would have thought that the emergence of sails on ships would have been a massive
        technological advancement. However, it seems that the Kvalsund ships were "a transition between a rowing ship and the typical Viking long ship, which combined rowing and sailing." \citep{NORDEIDE2020102192}.

        \pagebreak
        \bibliographystyle{apalike}
        \bibliography{myrefs}
        \cite{NORDEIDE2020102192}
\end{document}
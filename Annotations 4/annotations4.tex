\documentclass[a4paper]{article}
\usepackage[utf8]{inputenc}
\usepackage[english]{babel} 
\usepackage{hyperref} 
\usepackage{float}
\usepackage{amsmath} 
\usepackage{graphicx}
\usepackage[colorinlistoftodos]{todonotes} 
\usepackage{tikz}
\usepackage{pdfpages} 
\usepackage{setspace}
\usepackage{listings}
\usepackage[margin=0.5in]{geometry}
\usepackage{natbib}

\title {
	Annotations 4\\ Pantheon? What Pantheon? Concepts of a Family of Gods in Pre-Christian Scandinavian Religions
}

\author {
	\normalsize Sathvik Chinta\\\normalsize
    \normalsize HSTAM 370\\\normalsize
}

\date {
	\color{black} November 25, 2022
}

\doublespacing
\begin{document} \maketitle
    \setcitestyle{authoryear,open={(},close={)}}

        \textbf{a paragraph (150-250 words) summarizing the author's main argument and the evidence they use to support that argument}
        
        This work deals with the Pantheon of the Norse Gods, and common misconceptions that many people have about them in the modern day. 
        The article’s main goal is to consolidate information from multiple papers and aims to dispel the belief that “all Scandinavians 
        (including the Icelanders) believed in a pantheon of Æsir and Vanir gods who lived together in the same space…under the rulership 
        (and fatherhood) of Ódinn, each having his/her particular social or natural function…[and] that all dead heroes went to Valholl” 
        \citep[p.55]{TGunn}. Previously, many thought these beliefs had been constant across all Nordic and Germanic areas throughout 
        the while Viking period. The article argues that there were many factors, however, that influenced different groups beliefs 
        (such as geographic, social, and temporal factors).

        The articles begins by stating that the idea of a pantheon of gods emerged because of “religious systems”, and how many different 
        groups had different renditions of the gods. For instance, “There is even some dispute about which goddess was most closely 
        associated with Ódinn (Frigg, Sága, Freyja and others all vying for the title in different accounts)” \citep[p.57]{TGunn}. 
        The topic of Odin being the ruler of all the gods came under much scrutiny as well. Different areas had different levels of 
        reverence to different gods. One of the most popular gods was Tórr,  who was very prominent in place names and runes in Norway 
        and Sweden. Many of Tórr’s myths have him needing little to no assistance from other being to complete his tasks, and with 
        him “Never personally introducing himself directly as a son of Ódinn” \citep[p.65]{TGunn}. In other locations, Freyr has 
        also been though of as an “all-purpose” god similar to Tórr.

        The article ends by stating that there is a valid argument to indicate that people in the Viking age did not view the Norse 
        Pantheon as we do now, and that “there is good reason for questioning the idea that most people in Scandinavia ever imagined 
        the existence of a general family pantheon of gods… and that we instead consider a form of henotheism” \citep[p.69]{TGunn}.


        \textbf{a brief paragraph (100-150 words) that analyzes, with at least one example, how the reading changes your understanding of another reading or lecture topic in this or another course you have taken or are taking at UW.}

        The most surprising part of this article to me was the revelation that Ódinn was not as popular as initially believed. We learned 
        in class that many Nordic regions had names that were influenced by Ódinn, leading me to believe that Ódinn was considered intergral 
        to the pantheon of the gods in many places. However, with the emergence of Tórr as a much more prevalent diety (and the fact that 
        he never stated he was related to Ódinn/that he was part of the pantheon), it instead paints a picture of multiple gods each having 
        their own pockets of influence throughout Nordic countries rather than a central pantheon that everyone believed in. 

        \pagebreak
        \bibliographystyle{apalike}
        \bibliography{myrefs}
        \cite{TGunn}
\end{document}
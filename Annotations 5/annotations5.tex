\documentclass[a4paper]{article}
\usepackage[utf8]{inputenc}
\usepackage[english]{babel} 
\usepackage{hyperref} 
\usepackage{float}
\usepackage{amsmath} 
\usepackage{graphicx}
\usepackage[colorinlistoftodos]{todonotes} 
\usepackage{tikz}
\usepackage{pdfpages} 
\usepackage{setspace}
\usepackage{listings}
\usepackage[margin=0.5in]{geometry}
\usepackage{natbib}

\title {
	Annotations 5\\ Islam and Scandinavia during the Viking Age
}

\author {
	\normalsize Sathvik Chinta\\\normalsize
    \normalsize HSTAM 370\\\normalsize
}

\date {
	\color{black} December 8, 2022
}

\doublespacing
\begin{document} \maketitle
    \setcitestyle{authoryear,open={(},close={)}}

        \textbf{a paragraph (150-250 words) summarizing the author's main argument and the evidence they use to support that argument}
        
        This article poses a question: did Islam have a prominent role in Scandinavian society during the Viking Age? With the onset
        of Christianity and its battel with the old Nordic religions, the author argues that it is reasonable to question whether 
        Islam had a horse in the race as well. After all, Arab traders were known to frequent the Nordic region, with many objects 
        of Arab descent being found as artifacts. The author argues that the flow of objects (trade goods) may also carry flows of 
        religious ideals as well. However, the presence of religious objects does not necessarily indicate religion. 
        “An object may have a religious meaning in one context, but a different one in another. Therefore the context of 
        such an object must be studied at the ‘dispatcher’ end as well as the ‘recipient’ end.” \citep[p.39]{EMikk}. The 
        author begins by breaking down all the different items of Arabic descent found throughout Scandinavia such as Finger-rings, 
        clothing items, vessels, and more. One of the most interesting aspects of these objects are the graffiti inscriptions. 
        On coins they show “Thor's hammers and Christian crosses scratched across the quotations form the Quran” \citep[p.49]{EMikk}. 
        This, along with other evidence, indicates that the Vikings were aware of Islam as a religion and showed that people during 
        the Viking Age attempted to dispel the Islamic messages. The author concludes that there is enough evidence to support 
        that there was a lot of contact between the Arab and Scandinavian worlds, and that Islam may well have played apart in the 
        “’religious conflict’ in Viking Age Scandinavia” \citep[p.50]{EMikk}.


        \textbf{a brief paragraph (100-150 words) that analyzes, with at least one example, how the reading changes your understanding of another reading or lecture topic in this or another course you have taken or are taking at UW.}

        This article was particularly interesting. I took a class last quarter where we were taught the history of India 
        (with a heavy focus on Islam History and Hinduism History) through film. We saw that there was heavy influence of Islam in 
        India as well as in the Middle East, but the film makers made it seem like there were no attempts to make the religion 
        spread north towards the Scandinavian countries. However, this article shows that there were very prominent trade routes 
        towards the Nordic countries and provides a compelling argument to show that Islam may have played a part in the religious 
        conflict in the Viking Age. This is also opposed to what the filmmakers claimed, which was a patriotic recounting of the 
        holy wars saying that they were undefeated in their conquests. 

        \pagebreak
        \bibliographystyle{apalike}
        \bibliography{myrefs}
        \cite{EMikk}
\end{document}